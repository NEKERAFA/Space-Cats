\documentclass[12pt, spanish, a4paper]{article}

\usepackage[spanish]{babel}
\usepackage{anysize}
\usepackage[utf8]{inputenc}
\usepackage{amsmath}
\usepackage{tabularx}
\usepackage{lipsum}

% Configuration
\marginsize{2.5cm}{2.5cm}{2.25cm}{2.25cm}

\title{Space Cats \\ Documento de diseño del videojuego}
\date{\today}
\author{Rafael Alcalde Azpiazu}

\begin{document}
	
	% Title
	\pagenumbering{gobble}
	\maketitle
	\newpage
	
	% Contents
	\tableofcontents
	\newpage
	
	% Document
	\pagenumbering{arabic}
	
	\section{Historial de cambios}
	
	\subsection{Revisión 1.0}
	
	\paragraph{26 Febrero 2017} He creado un borrador del documento.
	
	\subsection{Versión 1.1}
	
	\paragraph{22 Abril 2017} He ido editado el documento, añadiendo la descripción y todas las ideas y consideración de cara a la creación de este videojuego. Espero que quede bien explicado.
	
	\subsection{Versión 1.2}
	
	\paragraph{23 Abril 2017} ** En trabajo **
	
	\newpage
	
	\section{Descripción}
	
	Space Cats es un videojuego Shoot'em Up espacial en un universo de fantasías con animales antropomorfos. Controlas una nave que está pilotada por un gato y tu misión es ir terminando una serie de niveles que van apareciendo conforme avances. \\
	
	Cada nivel constará de una serie de oleadas de naves enemigas, en donde la idea es que el jugador las destruya mediante diferentes armas que tiene su nave. Los enemigos darán puntos cada vez que sean destruidos. Una vez hecho, podrán arrojar algún power-up que hará cambiar el armamento de la nave, darle más vida o darle un multiplicador de puntos. \\
	
	En algunos niveles especiales, al final de estos podrá aparecer un enemigo más grande y con más vida, al que se le llamará Jefe del nivel, el cual arrojar más puntuación y mejoras para la nave una vez se destruya.
	
	\subsection{Consideraciones}
	
	Se que es un proyecto pequeño, pero me gustaría hacerlo bien desde el principio, con buenas prácticas y bien documentado. Se que va a ser un trabajo bastante grande para lo que el juego requiere, pero por lo menos aprendo a llevar un cierto orden y una metodología. \\
	
	He pensado que el producto software va a estar a disposición de quien haga falta (tanto documentación como código) en un repositorio de GitHub. En principio estará mantenido como un proyecto de software libre. Es la mejor forma de trabajar, a pesar de que de momento solo lo esté haciendo yo. Cabe destacar que GitHub también me puede servir de Back-up. \\
	
	Por último, que no estoy muy seguro, me gustaría que el juego sea multiplataforma. En un primer momento para PC (estaba pensando en usar el motor gráfico de Löve, ya que Lua es el lenguaje que mejor conozco).
	
	\section{Mundo}
	
	En este apartado hablaré sobre el mundo y elementos de juego de Space Cats, así como de las mecánicas que usará y como funcionará.
	
	\subsection{Descripción}
	
	\lipsum[10]
	
	\subsection{Gameplay}

	\lipsum[11]
	
	\subsection{Game Engine}
	
	\lipsum[12]
	
	\subsection{Layout e interfaz}
	
	\lipsum[13]
	
	\subsection{Escenarios}
	
	\lipsum[14]
	
	\subsection{Obstáculos}
	
	\lipsum[15]
	
	\subsection{Armas}
	
	\lipsum[16]
	
	\subsection{Power-ups}
	
	\lipsum[17]
	
	\section{Naves}
	
	\lipsum[18]
	
	\subsection{Naves del jugador}
	
	\lipsum[19]
	
	\subsection{Naves enemigas}
	
	\lipsum[20]
	
	\section{Arte}
	
	En este apartado hablaré sobre todo lo que tenga que ver con el diseño visual y sonoro del juego. Intentaré incluir las referencias a los archivos que se usarán, tanto creados por mi como todos aquellos que recoja de Internet.
	
	\subsection{Descripción}
	
	\lipsum[22]
	
	\subsection{Música}
	
	\lipsum[23]
	
	\subsection{Efectos de sonido}
	
	\lipsum[24]
	
	\subsection{Elementos}
	
	\lipsum[25]
	
	\subsection{Fondos de los escenarios}
	
	\lipsum[26]
	
	\subsection{Objectos}
	
	\lipsum[27]
	
\end{document}
